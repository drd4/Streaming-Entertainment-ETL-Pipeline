\documentclass[11pt,]{article}
\usepackage[]{mathpazo}
\usepackage{amssymb,amsmath}
\usepackage{ifxetex,ifluatex}
\usepackage{fixltx2e} % provides \textsubscript
\ifnum 0\ifxetex 1\fi\ifluatex 1\fi=0 % if pdftex
  \usepackage[T1]{fontenc}
  \usepackage[utf8]{inputenc}
\else % if luatex or xelatex
  \ifxetex
    \usepackage{mathspec}
  \else
    \usepackage{fontspec}
  \fi
  \defaultfontfeatures{Ligatures=TeX,Scale=MatchLowercase}
\fi
% use upquote if available, for straight quotes in verbatim environments
\IfFileExists{upquote.sty}{\usepackage{upquote}}{}
% use microtype if available
\IfFileExists{microtype.sty}{%
\usepackage[]{microtype}
\UseMicrotypeSet[protrusion]{basicmath} % disable protrusion for tt fonts
}{}
\PassOptionsToPackage{hyphens}{url} % url is loaded by hyperref
\usepackage[unicode=true]{hyperref}
\hypersetup{
            pdftitle={Location is Everything, Right?},
            pdfauthor={Daniel Dasgupta drd2141},
            pdfborder={0 0 0},
            breaklinks=true}
\urlstyle{same}  % don't use monospace font for urls
\usepackage[margin=1in]{geometry}
\usepackage{color}
\usepackage{fancyvrb}
\newcommand{\VerbBar}{|}
\newcommand{\VERB}{\Verb[commandchars=\\\{\}]}
\DefineVerbatimEnvironment{Highlighting}{Verbatim}{commandchars=\\\{\}}
% Add ',fontsize=\small' for more characters per line
\usepackage{framed}
\definecolor{shadecolor}{RGB}{248,248,248}
\newenvironment{Shaded}{\begin{snugshade}}{\end{snugshade}}
\newcommand{\KeywordTok}[1]{\textcolor[rgb]{0.13,0.29,0.53}{\textbf{#1}}}
\newcommand{\DataTypeTok}[1]{\textcolor[rgb]{0.13,0.29,0.53}{#1}}
\newcommand{\DecValTok}[1]{\textcolor[rgb]{0.00,0.00,0.81}{#1}}
\newcommand{\BaseNTok}[1]{\textcolor[rgb]{0.00,0.00,0.81}{#1}}
\newcommand{\FloatTok}[1]{\textcolor[rgb]{0.00,0.00,0.81}{#1}}
\newcommand{\ConstantTok}[1]{\textcolor[rgb]{0.00,0.00,0.00}{#1}}
\newcommand{\CharTok}[1]{\textcolor[rgb]{0.31,0.60,0.02}{#1}}
\newcommand{\SpecialCharTok}[1]{\textcolor[rgb]{0.00,0.00,0.00}{#1}}
\newcommand{\StringTok}[1]{\textcolor[rgb]{0.31,0.60,0.02}{#1}}
\newcommand{\VerbatimStringTok}[1]{\textcolor[rgb]{0.31,0.60,0.02}{#1}}
\newcommand{\SpecialStringTok}[1]{\textcolor[rgb]{0.31,0.60,0.02}{#1}}
\newcommand{\ImportTok}[1]{#1}
\newcommand{\CommentTok}[1]{\textcolor[rgb]{0.56,0.35,0.01}{\textit{#1}}}
\newcommand{\DocumentationTok}[1]{\textcolor[rgb]{0.56,0.35,0.01}{\textbf{\textit{#1}}}}
\newcommand{\AnnotationTok}[1]{\textcolor[rgb]{0.56,0.35,0.01}{\textbf{\textit{#1}}}}
\newcommand{\CommentVarTok}[1]{\textcolor[rgb]{0.56,0.35,0.01}{\textbf{\textit{#1}}}}
\newcommand{\OtherTok}[1]{\textcolor[rgb]{0.56,0.35,0.01}{#1}}
\newcommand{\FunctionTok}[1]{\textcolor[rgb]{0.00,0.00,0.00}{#1}}
\newcommand{\VariableTok}[1]{\textcolor[rgb]{0.00,0.00,0.00}{#1}}
\newcommand{\ControlFlowTok}[1]{\textcolor[rgb]{0.13,0.29,0.53}{\textbf{#1}}}
\newcommand{\OperatorTok}[1]{\textcolor[rgb]{0.81,0.36,0.00}{\textbf{#1}}}
\newcommand{\BuiltInTok}[1]{#1}
\newcommand{\ExtensionTok}[1]{#1}
\newcommand{\PreprocessorTok}[1]{\textcolor[rgb]{0.56,0.35,0.01}{\textit{#1}}}
\newcommand{\AttributeTok}[1]{\textcolor[rgb]{0.77,0.63,0.00}{#1}}
\newcommand{\RegionMarkerTok}[1]{#1}
\newcommand{\InformationTok}[1]{\textcolor[rgb]{0.56,0.35,0.01}{\textbf{\textit{#1}}}}
\newcommand{\WarningTok}[1]{\textcolor[rgb]{0.56,0.35,0.01}{\textbf{\textit{#1}}}}
\newcommand{\AlertTok}[1]{\textcolor[rgb]{0.94,0.16,0.16}{#1}}
\newcommand{\ErrorTok}[1]{\textcolor[rgb]{0.64,0.00,0.00}{\textbf{#1}}}
\newcommand{\NormalTok}[1]{#1}
\usepackage{longtable,booktabs}
% Fix footnotes in tables (requires footnote package)
\IfFileExists{footnote.sty}{\usepackage{footnote}\makesavenoteenv{long table}}{}
\usepackage{graphicx,grffile}
\makeatletter
\def\maxwidth{\ifdim\Gin@nat@width>\linewidth\linewidth\else\Gin@nat@width\fi}
\def\maxheight{\ifdim\Gin@nat@height>\textheight\textheight\else\Gin@nat@height\fi}
\makeatother
% Scale images if necessary, so that they will not overflow the page
% margins by default, and it is still possible to overwrite the defaults
% using explicit options in \includegraphics[width, height, ...]{}
\setkeys{Gin}{width=\maxwidth,height=\maxheight,keepaspectratio}
\IfFileExists{parskip.sty}{%
\usepackage{parskip}
}{% else
\setlength{\parindent}{0pt}
\setlength{\parskip}{6pt plus 2pt minus 1pt}
}
\setlength{\emergencystretch}{3em}  % prevent overfull lines
\providecommand{\tightlist}{%
  \setlength{\itemsep}{0pt}\setlength{\parskip}{0pt}}
\setcounter{secnumdepth}{5}
% Redefines (sub)paragraphs to behave more like sections
\ifx\paragraph\undefined\else
\let\oldparagraph\paragraph
\renewcommand{\paragraph}[1]{\oldparagraph{#1}\mbox{}}
\fi
\ifx\subparagraph\undefined\else
\let\oldsubparagraph\subparagraph
\renewcommand{\subparagraph}[1]{\oldsubparagraph{#1}\mbox{}}
\fi

% set default figure placement to htbp
\makeatletter
\def\fps@figure{htbp}
\makeatother


\title{Location is Everything, Right?}
\author{Daniel Dasgupta drd2141}
\date{November 30, 2020}

\begin{document}
\maketitle
\begin{abstract}
Optional summary goes here.
\end{abstract}

\section{Initial exploration}\label{initial-exploration}

During my initial exploration, I evaluated the differences between the
two provided datasets to determine what needed to be done in my
analysis. The analysis dataset (which I named `data'), has 39,527
observations with 91 variables. The scoring dataset, had 9,882
observations with 90 variables: the difference between the two datasets
is the inclusion of price in the analysis dataset.

I wanted to take a closer look at the contents of the dataset which
variables had missing values or outliers. First, I evaluated the
variable types to determine if I needed to convert specific variables to
character strings or factors to combine the datasets later on in my
analysis. In the scoring dataset, I found that zipcode was stored as an
integer but in the data, zipcode is a character. I then converted the
zipcode in the scoring dataset to a character to combine the two
datasets.

I combinded the datasets to have an overview of which variables had
missing values or N/A's. To do this, I created `missing\_col\_total' to
examine which variables had missing data.

\begin{Shaded}
\begin{Highlighting}[]
\NormalTok{## READ IN DATA HERE}
\NormalTok{data =}\StringTok{ }\KeywordTok{read.csv}\NormalTok{(}\StringTok{'analysisData.csv'}\NormalTok{)}
\NormalTok{scoring =}\StringTok{ }\KeywordTok{read.csv}\NormalTok{(}\StringTok{'scoringData.csv'}\NormalTok{)}
\NormalTok{## INITIAL EXPLORATION STARTS HERE}
\CommentTok{#Let's take a look at the variable types in both datasets}
\KeywordTok{str}\NormalTok{(data)}
\KeywordTok{str}\NormalTok{(scoring) }\CommentTok{#zipcode is an integer}

\NormalTok{scoring}\OperatorTok{$}\NormalTok{zipcode <-}\StringTok{ }\KeywordTok{as.character}\NormalTok{(scoring}\OperatorTok{$}\NormalTok{zipcode) }\CommentTok{#by converting zipcode into a character, I am now able to combine the datasets}

\CommentTok{#I created a new dataframe, total, by binding the scoring and data datasets.}

\NormalTok{total <-}\StringTok{ }\KeywordTok{bind_rows}\NormalTok{(data, scoring)}

\NormalTok{missing_col_total =}\StringTok{ }\KeywordTok{colSums}\NormalTok{(}\KeywordTok{is.na}\NormalTok{(total)); missing_col_total[missing_col_total }\OperatorTok{>}\StringTok{ }\DecValTok{0}\NormalTok{]}
\CommentTok{# this }


\NormalTok{...}
\NormalTok{## INITIAL EXPLORATION }\RegionMarkerTok{END}\NormalTok{S HERE}
\end{Highlighting}
\end{Shaded}

\section{Models and Feature
Selection}\label{models-and-feature-selection}

Write a paragraph about each model and feature selection you tried, why
you used it, and what it told you. Make it clear when a feature
selection technique you are describing is being used for one model and
not another. Especially make sure to highlight which method you used for
your final (i.e.~best performing) submission.

Write this section in your own words, since code will go into your code
submission. In the R code you submit, be sure to mark what part of the
code was used for initial exploration by using comments, for example:

\begin{Shaded}
\begin{Highlighting}[]
\NormalTok{## MODELING AND FEATURE SELECTION STARTS HERE}
\NormalTok{## FEATURE SELECTION 1: SUBSET FORWARD SEARCH STARTS HERE}
\NormalTok{…}
\NormalTok{## FEATURE SELECTION 1: SUBSET FORWARD SEARCH }\RegionMarkerTok{END}\NormalTok{S HERE}

\NormalTok{## MODEL 1: LINEAR REGRESSION WITH FEATURES FROM FEATURE SELECTION 1 STARTS HERE}
\NormalTok{…}
\NormalTok{## MODEL 1: LINEAR REGRESSION WITH FEATURES FROM FEATURE SELECTION 1 }\RegionMarkerTok{END}\NormalTok{S HERE}

\NormalTok{## MODEL 2: DECISION TREE WITH FEATURES FROM FEATURE SELECTION 1 STARTS HERE}
\NormalTok{…}
\NormalTok{## MODEL 2: DECISION TREE WITH FEATURES FROM FEATURE SELECTION 1 }\RegionMarkerTok{END}\NormalTok{S HERE}
\NormalTok{...}

\NormalTok{## MODELING AND FEATURE SELECTION }\RegionMarkerTok{END}\NormalTok{S HERE}
\end{Highlighting}
\end{Shaded}

\section{Model Comparison}\label{model-comparison}

State which combination of model and feature selection worked best, and
what your Kaggle score was. It may help to make a table that describes
what you know about each model, it is ok to leave blanks here if you
don't know something or it doesn't apply, for example:

\begin{longtable}[]{@{}lllll@{}}
\toprule
\begin{minipage}[b]{0.20\columnwidth}\raggedright\strut
Model from previous section\strut
\end{minipage} & \begin{minipage}[b]{0.14\columnwidth}\raggedright\strut
RMSE on training data\strut
\end{minipage} & \begin{minipage}[b]{0.20\columnwidth}\raggedright\strut
RMSE for test data holdout or CV\strut
\end{minipage} & \begin{minipage}[b]{0.10\columnwidth}\raggedright\strut
RMSE on Kaggle\strut
\end{minipage} & \begin{minipage}[b]{0.22\columnwidth}\raggedright\strut
Other notes\strut
\end{minipage}\tabularnewline
\midrule
\endhead
\begin{minipage}[t]{0.20\columnwidth}\raggedright\strut
Model 1: Linear Regression\strut
\end{minipage} & \begin{minipage}[t]{0.14\columnwidth}\raggedright\strut
90.23879\strut
\end{minipage} & \begin{minipage}[t]{0.20\columnwidth}\raggedright\strut
92.9834\strut
\end{minipage} & \begin{minipage}[t]{0.10\columnwidth}\raggedright\strut
89.37893\strut
\end{minipage} & \begin{minipage}[t]{0.22\columnwidth}\raggedright\strut
\strut
\end{minipage}\tabularnewline
\begin{minipage}[t]{0.20\columnwidth}\raggedright\strut
Model 2: Decision tree\strut
\end{minipage} & \begin{minipage}[t]{0.14\columnwidth}\raggedright\strut
86.9834\strut
\end{minipage} & \begin{minipage}[t]{0.20\columnwidth}\raggedright\strut
\strut
\end{minipage} & \begin{minipage}[t]{0.10\columnwidth}\raggedright\strut
86.3498\strut
\end{minipage} & \begin{minipage}[t]{0.22\columnwidth}\raggedright\strut
Used features selected for model 1\strut
\end{minipage}\tabularnewline
\begin{minipage}[t]{0.20\columnwidth}\raggedright\strut
\strut
\end{minipage} & \begin{minipage}[t]{0.14\columnwidth}\raggedright\strut
\strut
\end{minipage} & \begin{minipage}[t]{0.20\columnwidth}\raggedright\strut
\strut
\end{minipage} & \begin{minipage}[t]{0.10\columnwidth}\raggedright\strut
\strut
\end{minipage} & \begin{minipage}[t]{0.22\columnwidth}\raggedright\strut
\strut
\end{minipage}\tabularnewline
\begin{minipage}[t]{0.20\columnwidth}\raggedright\strut
Model 3: CV-tuned Random forest\strut
\end{minipage} & \begin{minipage}[t]{0.14\columnwidth}\raggedright\strut
62.9411\strut
\end{minipage} & \begin{minipage}[t]{0.20\columnwidth}\raggedright\strut
73.1331\strut
\end{minipage} & \begin{minipage}[t]{0.10\columnwidth}\raggedright\strut
75.1234\strut
\end{minipage} & \begin{minipage}[t]{0.22\columnwidth}\raggedright\strut
Final Kaggle RMSE: 78.1223\strut
\end{minipage}\tabularnewline
\bottomrule
\end{longtable}

\section{Discussion}\label{discussion}

Describe and discuss the analysis that gave you the best results. Why do
you think this was the top performing model for the `cleaned-up' data?
Can you use the model to make any inferences about the relationships
between predictor variables and Airbnb prices?

\section{Future directions}\label{future-directions}

What would you do for this project if you had more time?

\section{Appendix}\label{appendix}

This section is optional, but use it to describe alternative approaches
that did not lead to your best model, but nevertheless seemed promising.

\end{document}
